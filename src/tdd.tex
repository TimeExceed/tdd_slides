\documentclass[lualatex]{beamer}
\usepackage[english]{babel}
\usepackage[T1]{fontenc}
\usepackage{mathptmx}
\usepackage[scaled=.90]{helvet}
\usepackage{courier}
\usepackage{graphicx}
\usepackage{listings}

\setbeameroption{show notes}
\usetheme{Madrid}
\lstset{basicstyle=\footnotesize\ttfamily,language={C++}}

\title[TDD]{A Short Introduction to\\Test-Driven Development}
  \author{Taoda}
  \institute{Aliyun}

\begin{document}

\begin{frame}
\titlepage
\end{frame}

\begin{frame}
  \frametitle{Outline}
  \tableofcontents
\end{frame}

\note{
  This talk introduces software engineering knowledge of the last century.
  In last century, SE is an area of management.
  Since the millineum, this area rapidly turns to formal methods.
  They are topics of other talks.
}

\section{Background}

\subsection[steps]{development elements}

\begin{frame}
  \frametitle{Development elements}

  \begin{block}{~}
    \begin{enumerate}
    \item requirement analyze
    \item design
    \item code
    \item test
    \item release/deploy
    \end{enumerate}
  \end{block}
  \note[item]{
    In waterfall model, people do these 5 steps sequentially. If every step can be done perfectly and there is a very limited group of human resources, this is the most efficient model in theory.
  }
  \note[item]{
    In big company, people usually focus on their own areas.
    To exploit these people, V model rises.
    However, both V model and waterfall model work only if these 5 steps are perfectly done.
    This is usually not true in internet companies.
  }
\end{frame}

\begin{frame}
  \frametitle{CMMI}
  \note[item]{
    \scriptsize
    CMU's mature models for many areas.
    The most famous one of them is software mature model, aka, CMM-Software.
    After many years application, CMU realized that people have difficulties in distinguishing her models.
    So, CMU integrated her models based on CMM-Software.
    That was CMMI.
  }

  \begin{block}{levels}
    \begin{enumerate}
    \item initial
      \note[item]{
        \scriptsize
        Initial level considers success of software projects as windfall.
      }
    \item managed: project-level 
      \note[item]{
        \scriptsize
        In managed level, project success will be achieved repeatedly.
        It recognizes 6 important elements, not 6 steps.
        These elements are must-do.
        CMMI says nothing about how-to-do.
        \begin{itemize}
        \scriptsize
        \item REQM: 
          Analyze requirements, pass them to devs and testers, and track them.
        \item PP: 
          What, Who, and When.
        \item PMC:
          How things go on?
        \item SAM:
          How to assure release quality and deadlines of 2nd-party or 3rd-party libraries,
          and how to track them and push them.
          There are usually two ways:
          i) making contracts, or 
          2) everyone can access and modify everything.
        \item PPQA:
          This element is more than testing.
          It also includes deployment processes.
        \item CM:
          It includes how to store codes, documents, delivery binaries or packages and flags/environments and how to deploy them.
        \end{itemize}
      }
      \begin{itemize}
      \item REQuirement Management
      \item Project Planning
      \item Project Monitoring and Control
      \item Supply Agreement Management 
      \item Product and Process Quality Assurance
      \item Configuration Management
      \end{itemize}
    \item defined: organization-wide
    \item quantitatively managed: organizational performance 
    \item optimized: risk and innovation 
      \note[item]{
        \scriptsize
        These organizational levels are unrelated to topics about this talk.
      }
    \end{enumerate}
  \end{block}
\end{frame}

\section[Testing Framework]{Python Testing Framework: An Example}

\subsection{A To-Do List}

\begin{frame}
    \frametitle{What is a testing frame supposed to do?}

    \begin{block}{To-Do}
        \begin{itemize}
            \item Invoke test method
            \item Invoke setUp first 
            \item Invoke tearDown afterward 
            \item Invoke tearDown even if the test method fails 
            \item Run multiple tests 
            \item Report collected results 
        \end{itemize}
    \end{block}
\end{frame}

\subsection{test method}

\begin{frame}[fragile,t]
    \frametitle{Invoke test method}

    \begin{columns}[t]
        \small
        \column{.45\textwidth}
        \begin{onlyenv}<1-2>
            \begin{block}{test cases}
                \begin{lstlisting}[language=Python,columns=fullflexible]
test = WasRun()
assert not test.wasRun
test.testMethod()
assert test.wasRun
                \end{lstlisting}
            \end{block}
        \end{onlyenv}

        \begin{onlyenv}<3>
            \begin{block}{test cases}
                \begin{lstlisting}[language=Python,columns=fullflexible]
test = WasRun()
assert not test.wasRun
test.run()
assert test.wasRun
                \end{lstlisting}
            \end{block}
        \end{onlyenv}

        \begin{onlyenv}<4->
            \begin{block}{test cases}
                \begin{lstlisting}[language=Python,columns=fullflexible]
test = WasRun("testMethod")
assert not test.wasRun
test.run()
assert test.wasRun
                \end{lstlisting}
            \end{block}
        \end{onlyenv}

        \column{.45\textwidth}
        \begin{onlyenv}<2>
            \begin{block}{functional codes}
                \begin{lstlisting}[language=Python,columns=fullflexible]
class WasRun:
  def __init__(self):
    self.wasRun = 0
  def testMethod(self):
    self.wasRun = 1
                \end{lstlisting}
            \end{block}
        \end{onlyenv}
        \note<2>{
          The 1st to-do is fulfilled, though it is hacky.
          What to do then?
          Not a next feature.
          Refactory!
        }
        \begin{onlyenv}<3-4>
            \begin{block}{functional codes}
                \begin{lstlisting}[language=Python,columns=fullflexible]
class WasRun:
  def __init__(self):
    self.wasRun = 0
  def testMethod(self):
    self.wasRun = 1
  def run(self):
    self.testMethod()
                \end{lstlisting}
            \end{block}
        \end{onlyenv}
        
        \begin{onlyenv}<5>
            \begin{block}{functional codes}
                \begin{lstlisting}[language=Python,columns=fullflexible]
class WasRun:
  def __init__(self, name):
    self.wasRun = 0
    self.name = name
  def testMethod(self):
    self.wasRun = 1
  def run(self):
    method=getattr(self, self.name)
    method()
                \end{lstlisting}
            \end{block}
        \end{onlyenv}
        \note[item]<5>{
          WasRun.run() looks quite general.
          It is a good idea to extract a base class.
        }
    \end{columns}        
\end{frame}

\begin{frame}[fragile,t]
    \frametitle{Invoke test method}

    \begin{columns}[t]
        \small
        \column{.45\textwidth}
        \begin{onlyenv}<1>
            \begin{block}{functional codes}
                \begin{lstlisting}[language=Python,columns=fullflexible]
class TestCase:
  pass
                \end{lstlisting}
            \end{block}
        \end{onlyenv}
        \note<1>{
          Do a little thing. Make sure UTs pass.
        }

        \begin{onlyenv}<2>
            \begin{block}{functional codes}
                \begin{lstlisting}[language=Python,columns=fullflexible]
class TestCase:
  def __init__(self, name):
    self.name = name
                \end{lstlisting}
            \end{block}
        \end{onlyenv}

        \begin{onlyenv}<3->
            \begin{block}{functional codes}
                \begin{lstlisting}[language=Python,columns=fullflexible]
class TestCase:
  def __init__(self, name):
    self.name = name
  def run(self):
    method=getattr(self, self.name)
    method()
                \end{lstlisting}
            \end{block}
        \end{onlyenv}

        \column{.45\textwidth}
        \begin{onlyenv}<1>
            \begin{block}{functional codes}
                \begin{lstlisting}[language=Python,columns=fullflexible]
class WasRun(TestCase):
  def __init__(self, name):
    self.wasRun = 0
    self.name = name
  def testMethod(self):
    self.wasRun = 1
  def run(self):
    method=getattr(self, self.name)
    method()
                \end{lstlisting}
            \end{block}
        \end{onlyenv}

        \begin{onlyenv}<2>
            \begin{block}{functional codes}
                \begin{lstlisting}[language=Python,columns=fullflexible]
class WasRun(TestCase):
  def __init__(self, name):
    TestCase.__init__(self, name)
    self.wasRun = 0
  def testMethod(self):
    self.wasRun = 1
  def run(self):
    method=getattr(self, self.name)
    method()
                \end{lstlisting}
            \end{block}
        \end{onlyenv}

        \begin{onlyenv}<3>
            \begin{block}{functional codes}
                \begin{lstlisting}[language=Python,columns=fullflexible]
class WasRun(TestCase):
  def __init__(self, name):
    TestCase.__init__(self, name)
    self.wasRun = 0
  def testMethod(self):
    self.wasRun = 1
                \end{lstlisting}
            \end{block}
        \end{onlyenv}

    \end{columns}
\end{frame}

\begin{frame}[fragile,t]
    \frametitle{Invoke test method}

    \begin{columns}[t]
        \small
        \column{.45\textwidth}
        \begin{onlyenv}<1->
            \begin{block}{functional codes}
                \begin{lstlisting}[language=Python,columns=fullflexible]
class TestCase:
  def __init__(self, name):
    self.name = name
  def run(self):
    method=getattr(self, self.name)
    method()
                \end{lstlisting}
            \end{block}
        \end{onlyenv}


        \column{.45\textwidth}
        \begin{onlyenv}<1>
            \begin{block}{test cases}
                \begin{lstlisting}[language=Python,columns=fullflexible]
test = WasRun()
assert not test.wasRun
test.run()
assert test.wasRun
                \end{lstlisting}
            \end{block}
        \end{onlyenv}

        \begin{onlyenv}<2>
            \begin{block}{test case}
                \begin{lstlisting}[language=Python,columns=fullflexible]
class TestCaseTest(TestCase):
  def testRunning(self):
    test=WasRun("testMethod")
    assert(not test.wasRun)
    test.run()
    assert(test.wasRun)
TestCaseTest("testRunning").run()
                \end{lstlisting}
            \end{block}
        \end{onlyenv}

    \end{columns}
\end{frame}

\begin{frame}
  \frametitle{review}

  \begin{block}{~}
    \begin{itemize}
    \item 
      Write test case first.
      Make it red.
    \item 
      Turn green asap, by even faking it or hardwiring it.
      Refine it then.
    \item 
      Small and steady paces.
    \item 
      Refine your test cases.
    \end{itemize}
    \end{block}
\end{frame}

\subsection{setUp}

\begin{frame}
    \frametitle{setUp}

    \begin{block}{To-Do}
        \begin{itemize}
            \item[$\surd$] Invoke test method
            \item Invoke setUp first 
            \item Invoke tearDown afterward 
            \item Invoke tearDown even if the test method fails 
            \item Run multiple tests 
            \item Report collected results 
        \end{itemize}
    \end{block}
    \note[item]{
      What to do next?
      If we can only do just one feature, who will benefit the product most?
    }
\end{frame}

\begin{frame}[fragile,t]
    \frametitle{setUp}

    \begin{columns}[t]
        \small
        \column{.45\textwidth}
        \begin{onlyenv}<1-4>
            \begin{block}{test case}
                \begin{lstlisting}[language=Python,columns=fullflexible]
class TestCaseTest(TestCase):
  def testRunning(self):
    test=WasRun("testMethod")
    assert(not test.wasRun)
    test.run()
    assert(test.wasRun)
  def testSetUp(self):
    test=WasRun("testMethod")
    test.run()
    assert(test.wasSetUp)
                \end{lstlisting}
            \end{block}
        \end{onlyenv}

        \begin{onlyenv}<5>
            \begin{block}{test case}
                \begin{lstlisting}[language=Python,columns=fullflexible]
class TestCaseTest(TestCase):
  def testRunning(self):
    test=WasRun("testMethod")
    test.run()
    assert(test.wasRun)
  def testSetUp(self):
    test=WasRun("testMethod")
    test.run()
    assert(test.wasSetUp)
                \end{lstlisting}
            \end{block}
        \end{onlyenv}

        \begin{onlyenv}<6>
            \begin{block}{test case}
                \begin{lstlisting}[language=Python,columns=fullflexible]
class TestCaseTest(TestCase):
  def setUp(self):
    self.test=WasRun("testMethod")
  def testRunning(self):
    self.test.run()
    assert(self.test.wasRun)
  def testSetUp(self):
    self.test.run()
    assert(self.test.wasSetUp)
                \end{lstlisting}
            \end{block}
        \end{onlyenv}

        \column{.45\textwidth}
        \begin{onlyenv}<2>
            \begin{block}{functional codes}
                \begin{lstlisting}[language=Python,columns=fullflexible]
class TestCase:
  def setUp(self):
    pass
  def run(self):
    self.setUp()
    method=getattr(self, self.name)
    method()
class WasRun:
  def setUp(self):
    self.wasSetUp = 1
                \end{lstlisting}
            \end{block}
        \end{onlyenv}

        \begin{onlyenv}<3>
            \begin{block}{functional codes}
                \begin{lstlisting}[language=Python,columns=fullflexible]
class WasRun(TestCase):
  def __init__(self,name):
    self.wasRun = 0
    TestCase.__init__(self, name)
  def testMethod(self):
    self.wasRun = 1
  def setUp(self):
    self.wasSetUp = 1
                \end{lstlisting}
            \end{block}
        \end{onlyenv}

        \begin{onlyenv}<4->
            \begin{block}{functional codes}
                \begin{lstlisting}[language=Python,columns=fullflexible]
class WasRun(TestCase):
  def __init__(self,name):
    TestCase.__init__(self, name)
  def testMethod(self):
    self.wasRun = 1
  def setUp(self):
    self.wasRun = 0
    self.wasSetUp = 1
                \end{lstlisting}
            \end{block}
        \end{onlyenv}

    \end{columns}
\end{frame}

\begin{frame}
    \frametitle{log string}

    \begin{block}{To-Do}
        \begin{itemize}
            \item[$\surd$] Invoke test method
            \item[$\surd$] Invoke setUp first 
            \item Invoke tearDown afterward 
            \item Invoke tearDown even if the test method fails 
            \item Run multiple tests 
            \item Report collected results 
            \item Log string in WasRun 
        \end{itemize}
    \end{block}
    \note[item]{
      We add a to-do item.
      We do ``log string in WasRun'' because this is a refactory.
      TDD supposes a project will be cut anytime.
      Even when it is cut unexpectedly, we deliver a working product with highly quality.
    }
\end{frame}

\begin{frame}[fragile,t]
    \frametitle{log string}

    \begin{columns}[t]
        \small
        \column{.45\textwidth}
        \begin{onlyenv}<1>
            \begin{block}{test case}
                \begin{lstlisting}[language=Python,columns=fullflexible]
class TestCaseTest(TestCase):
  def setUp(self):
    self.test=WasRun("testMethod")
  def testRunning(self):
    self.test.run()
    assert(self.test.wasRun)
  def testSetUp(self):
    self.test.run()
    assert(self.test.wasSetUp)
                \end{lstlisting}
            \end{block}
        \end{onlyenv}

        \begin{onlyenv}<2>
            \begin{block}{test case}
                \begin{lstlisting}[language=Python,columns=fullflexible]
class TestCaseTest(TestCase):
  def setUp(self):
    self.test=WasRun("testMethod")
  def testTemplateMethod(self):
    self.test.run()
    assert("setUp test "
      == self.test.log)
                \end{lstlisting}
            \end{block}
        \end{onlyenv}

        \begin{onlyenv}<3>
            \begin{block}{test case}
                \begin{lstlisting}[language=Python,columns=fullflexible]
class TestCaseTest(TestCase):
  def testTemplateMethod(self):
    test=WasRun("testMethod")
    test.run()
    assert("setUp test "
      == test.log)
                \end{lstlisting}
            \end{block}
        \end{onlyenv}

        \column{.45\textwidth}
        \begin{onlyenv}<1>
            \begin{block}{functional codes}
                \begin{lstlisting}[language=Python,columns=fullflexible]
class WasRun(TestCase):
  def __init__(self,name):
    TestCase.__init__(self, name)
  def testMethod(self):
    self.wasRun = 1
  def setUp(self):
    self.wasRun = 0
    self.wasSetUp = 1
                \end{lstlisting}
            \end{block}
        \end{onlyenv}

        \begin{onlyenv}<2>
            \begin{block}{functional codes}
                \begin{lstlisting}[language=Python,columns=fullflexible]
class WasRun(TestCase):
  def __init__(self,name):
    TestCase.__init__(self, name)
  def testMethod(self):
    self.wasRun = 1
    self.log = self.log + "test "
  def setUp(self):
    self.wasRun = 0
    self.wasSetUp = 1
    self.log = "setUp "
                \end{lstlisting}
            \end{block}
        \end{onlyenv}

        \begin{onlyenv}<3>
            \begin{block}{functional codes}
                \begin{lstlisting}[language=Python,columns=fullflexible]
class WasRun(TestCase):
  def __init__(self,name):
    TestCase.__init__(self, name)
  def testMethod(self):
    self.log = self.log + "test "
  def setUp(self):
    self.log = "setUp "
                \end{lstlisting}
            \end{block}
        \end{onlyenv}

    \end{columns}
\end{frame}

\subsection{tearDown}

\begin{frame}
    \frametitle{tearDown}

    \begin{block}{To-Do}
        \begin{itemize}
            \item[$\surd$] Invoke test method
            \item[$\surd$] Invoke setUp first 
            \item Invoke tearDown afterward 
            \item Invoke tearDown even if the test method fails 
            \item Run multiple tests 
            \item Report collected results 
            \item[$\surd$] Log string in WasRun 
        \end{itemize}
    \end{block}
\end{frame}

\begin{frame}[fragile,t]
    \frametitle{tearDown}

    \begin{columns}[t]
        \small
        \column{.45\textwidth}
        \begin{onlyenv}<1->
            \begin{block}{test case}
                \begin{lstlisting}[language=Python,columns=fullflexible]
class TestCaseTest(TestCase):
  def testTemplateMethod(self):
    test=WasRun("testMethod")
    test.run()
    assert("setUp test tearDown"
      == test.log)
                \end{lstlisting}
            \end{block}
        \end{onlyenv}

        \column{.45\textwidth}
        \begin{onlyenv}<2>
            \begin{block}{functional codes}
                \begin{lstlisting}[language=Python,columns=fullflexible]
class TestCase:
  def tearDown(self):
    pass
  def run(self):
    self.setUp()
    method=getattr(self, self.name)
    method()
    self.tearDown()
class WasRun(TestCase):
  def tearDown(self):
    self.log = self.log + "tearDown"
                \end{lstlisting}
            \end{block}
        \end{onlyenv}

    \end{columns}
\end{frame}

\begin{frame}
  \frametitle{review}

  \begin{block}{~}
    \begin{itemize}
    \item 
      To-do list is dynamic.
    \item 
      Order of tasks in to-do list is nothing about their importance.
    \end{itemize}
  \end{block}
\end{frame}

\subsection{report results}

\begin{frame}
  \frametitle{report results}

  \begin{block}{To-Do}
    \begin{itemize}
    \item[$\surd$] Invoke test method
    \item[$\surd$] Invoke setUp first 
    \item[$\surd$] Invoke tearDown afterward 
    \item Invoke tearDown even if the test method fails 
    \item Run multiple tests 
    \item Report collected results 
    \item[$\surd$] Log string in WasRun 
    \end{itemize}
  \end{block}
  \note{
    What is the most important?
    Reports.
    For now, we show nothing if all cases pass.
    This is quite ugly.
    The others are better-haves.
  }
\end{frame}

\begin{frame}[fragile,t]
    \frametitle{report results}

    \begin{columns}[t]
        \small
        \column{.45\textwidth}
        \begin{onlyenv}<1>
            \begin{block}{test case}
                \begin{lstlisting}[language=Python,columns=fullflexible]
class TestCaseTest(TestCase):
  def testResult(self):
    test=WasRun("testMethod")
    result = test.run()
    assert("1 run, 0 failed"
      == result.summary())
                \end{lstlisting}
            \end{block}
        \end{onlyenv}

        \begin{onlyenv}<2>
            \begin{block}{functional codes}
                \begin{lstlisting}[language=Python,columns=fullflexible]
class TestResult:
  def summary(self):
    return "1 run, 0 failed"
                \end{lstlisting}
            \end{block}
        \end{onlyenv}

        \begin{onlyenv}<3>
            \begin{block}{functional codes}
                \begin{lstlisting}[language=Python,columns=fullflexible]
class TestResult:
  def summary(self):
    return "%d run, 0 failed"
        % self.run
                \end{lstlisting}
            \end{block}
        \end{onlyenv}

        \begin{onlyenv}<4>
            \begin{block}{functional codes}
                \begin{lstlisting}[language=Python,columns=fullflexible]
class TestResult:
  def __init__(self):
    self.run = 0
  def summary(self):
    return "%d run, 0 failed"
        % self.run
                \end{lstlisting}
            \end{block}
        \end{onlyenv}

        \begin{onlyenv}<5->
            \begin{block}{functional codes}
                \begin{lstlisting}[language=Python,columns=fullflexible]
class TestResult:
  def __init__(self):
    self.run = 0
  def summary(self):
    return "%d run, 0 failed"
        % self.run
  def started(self):
    self.run=self.run+1
                \end{lstlisting}
            \end{block}
        \end{onlyenv}

        \column{.45\textwidth}
        \begin{onlyenv}<1>
            \begin{block}{functional codes}
                \begin{lstlisting}[language=Python,columns=fullflexible]
class TestResult:
  def summary(self):
    return "1 run, 0 failed"
class TestCase:
  def run(self):
    self.setUp()
    method=getattr(self, self.name)
    method()
    self.tearDown()
    return TestResult()
                \end{lstlisting}
            \end{block}
        \end{onlyenv}

        \begin{onlyenv}<2-5>
            \begin{block}{functional codes}
                \begin{lstlisting}[language=Python,columns=fullflexible]
class TestCase:
  def run(self):
    self.setUp()
    method=getattr(self, self.name)
    method()
    self.tearDown()
    return TestResult()
                \end{lstlisting}
            \end{block}
        \end{onlyenv}

        \begin{onlyenv}<6>
            \begin{block}{functional codes}
                \begin{lstlisting}[language=Python,columns=fullflexible]
class TestCase:
  def run(self):
    result=TestResult()
    result.started()
    self.setUp()
    method=getattr(self, self.name)
    method()
    self.tearDown()
    return result
                \end{lstlisting}
            \end{block}
        \end{onlyenv}
    \end{columns}
    \note<1>{
      Tests in TDD are actually documents, executable documents.
    }
\end{frame}

\begin{frame}
    \frametitle{failed results}

    \begin{block}{To-Do}
        \begin{itemize}
            \item[$\surd$] Invoke test method
            \item[$\surd$] Invoke setUp first 
            \item[$\surd$] Invoke tearDown afterward 
            \item Invoke tearDown even if the test method fails 
            \item Run multiple tests 
            \item Report collected results 
            \item[$\surd$] Log string in WasRun 
            \item Report failed results
        \end{itemize}
    \end{block}
\end{frame}

\begin{frame}[fragile,t]
    \frametitle{failed results}

    \begin{columns}[t]
        \small
        \column{.45\textwidth}
        \begin{onlyenv}<1>
            \begin{block}{test cases}
                \begin{lstlisting}[language=Python,columns=fullflexible]
class TestCaseTest(TestCase):
  def testFailedResult(self):
    test=WasRun("testBrokenMethod")
    result = test.run()
    assert("1 run, 1 failed"
      == result.summary())
                \end{lstlisting}
            \end{block}
        \end{onlyenv}

        \begin{onlyenv}<2->
            \begin{block}{test cases}
                \begin{lstlisting}[language=Python,columns=fullflexible]
class TestCaseTest(TestCase):
  def testFailedResult(self):
    test=WasRun("testBrokenMethod")
    result = test.run()
    assert("1 run, 1 failed"
      == result.summary())
  def testFailedResultFormat(self):
    result=TestResult()
    result.started()
    result.failed()
    assert("1 run, 1 failed"
      == result.summary())
                \end{lstlisting}
            \end{block}
        \end{onlyenv}

        \column{.45\textwidth}
        \begin{onlyenv}<1>
            \begin{block}{functional codes}
                \begin{lstlisting}[language=Python,columns=fullflexible]
class WasRun(TestCase):
  def testBrokenMethod(self):
    raise Exception
                \end{lstlisting}
            \end{block}
        \end{onlyenv}

        \begin{onlyenv}<2>
            \begin{block}{functional codes}
                \begin{lstlisting}[language=Python,columns=fullflexible]
class TestResult:
  def __init__(self):
    self.run=0
    self.failed=0
  def failed(self):
    self.failed=self.failed+1
  def summary(self):
    return "%d run, %d failed"
        % (self.run, self.failed)
                \end{lstlisting}
            \end{block}
        \end{onlyenv}

        \begin{onlyenv}<3>
            \begin{block}{functional codes}
                \begin{lstlisting}[language=Python,columns=fullflexible]
class TestCase:
  def run(self):
    result=TestResult()
    result.started()
    self.setUp()
    try:
      method=getattr(self,self.name)
      method()
    except:
      result.failed()
    self.tearDown()
    return result
                \end{lstlisting}
            \end{block}
        \end{onlyenv}
    \end{columns}
    \note<3>{
      These code reminds us that exceptions from setUp() and tearDown().
    }
\end{frame}

\begin{frame}
    \frametitle{failed results}

    \begin{block}{To-Do}
        \begin{itemize}
            \item[$\surd$] Invoke test method
            \item[$\surd$] Invoke setUp first 
            \item[$\surd$] Invoke tearDown afterward 
            \item Invoke tearDown even if the test method fails 
            \item Run multiple tests 
            \item[$\surd$] Report collected results 
            \item[$\surd$] Log string in WasRun 
            \item[$\surd$] Report failed results
            \item Catch and report setUp() and tearDown() exceptions
        \end{itemize}
    \end{block}
\end{frame}

\begin{frame}
    \frametitle{review}

    \begin{block}{~}
        \begin{itemize}
            \item 
                Tests are actually documents, in SE term, executable documents.
            \item 
                Notice a potential problem and note it on to-do list instead of addressing it immediately.
            \item 
                A test case, functional codes, a test case, \ldots\\
                v.s.\ test cases, functional codes, \ldots
                \note{
                  The key difference is feedbacking.
                }
        \end{itemize}
    \end{block}
\end{frame}

\subsection{test suite}

\begin{frame}
    \frametitle{test suite}

    \begin{block}{To-Do}
        \begin{itemize}
            \item[$\surd$] Invoke test method
            \item[$\surd$] Invoke setUp first 
            \item[$\surd$] Invoke tearDown afterward 
            \item Invoke tearDown even if the test method fails 
            \item Run multiple tests 
            \item[$\surd$] Report collected results 
            \item[$\surd$] Log string in WasRun 
            \item[$\surd$] Report failed results
            \item Catch and report setUp() and tearDown() errors
        \end{itemize}
    \end{block}
\end{frame}

\begin{frame}[fragile,t]
    \frametitle{test suite}

        \small
        \begin{onlyenv}<1>
            \begin{block}{test case}
                \begin{lstlisting}[language=Python,columns=fullflexible]
print TestCaseTest('testTemplateMethod').run().summary()                    
print TestCaseTest('testResult').run().summary()                    
print TestCaseTest('testFailedResultFormat').run().summary()                    
print TestCaseTest('testFailedResult').run().summary()                    
                \end{lstlisting}
            \end{block}
        \end{onlyenv}

\end{frame}

\begin{frame}[fragile,t]
    \frametitle{test suite}

    \begin{columns}[t]
        \small
        \column{.45\textwidth}
        \begin{onlyenv}<1->
            \begin{block}{test case}
                \begin{lstlisting}[language=Python,columns=fullflexible]
class TestCaseTest(TestCase):
  def testSuite(self):
    suite=TestSuite()
    suite.add(
      WasRun('testMethod'))
    suite.add(
      WasRun('testBrokenMethod'))
    result=suite.run()
    assert('2 run, 1 failed' 
      == result.summary())
                \end{lstlisting}
            \end{block}
        \end{onlyenv}

        \column{.45\textwidth}
        \begin{onlyenv}<2>
            \begin{block}{functional code}
                \begin{lstlisting}[language=Python,columns=fullflexible]
class TestSuite:
  def __init__(self):
    self.tests=[]
  def add(self, test):
    self.tests.append(test)
  def run(self):
    result=TestResult()
    for x in self.tests:
      x.run()
    return result
                \end{lstlisting}
            \end{block}
        \end{onlyenv}

        \begin{onlyenv}<3>
            \begin{block}{functional code}
                \begin{lstlisting}[language=Python,columns=fullflexible]
class TestSuite:
  def run(self):
    result=TestResult()
    for x in self.tests:
      result.merge(x.run())
    return result
class TestResult:
  def merge(self, result):
    self.run=self.run+result.run
    self.failed=self.failed+result.failed
                \end{lstlisting}
            \end{block}
        \end{onlyenv}

    \end{columns}
\end{frame}

\begin{frame}[fragile,t]
    \frametitle{test suite}

        \small
        \begin{onlyenv}<1>
            \begin{block}{test case}
                \begin{lstlisting}[language=Python,columns=fullflexible]
print TestCaseTest('testTemplateMethod').run().summary()                    
print TestCaseTest('testResult').run().summary()                    
print TestCaseTest('testFailedResultFormat').run().summary()                    
print TestCaseTest('testFailedResult').run().summary()                    
                \end{lstlisting}
            \end{block}
        \end{onlyenv}

        \begin{onlyenv}<2>
            \begin{block}{test case}
                \begin{lstlisting}[language=Python,columns=fullflexible]
suite=TestSuite()
suite.add(TestCaseTest('testTemplateMethod'))
suite.add(TestCaseTest('testResult'))
suite.add(TestCaseTest('testFailedResultFormat'))
suite.add(TestCaseTest('testFailedResult'))
result=suite.run()
print result.summary()
                \end{lstlisting}
            \end{block}
        \end{onlyenv}
\end{frame}

\begin{frame}
    \frametitle{test suite}

    \begin{block}{To-Do}
        \begin{itemize}
            \item[$\surd$] Invoke test method
            \item[$\surd$] Invoke setUp first 
            \item[$\surd$] Invoke tearDown afterward 
            \item Invoke tearDown even if the test method fails 
            \item Run multiple tests 
            \item[$\surd$] Report collected results 
            \item[$\surd$] Log string in WasRun 
            \item[$\surd$] Report failed results
            \item Catch and report setUp() and tearDown() errors
            \item Automatically create TestSuite() from a TestCase class
        \end{itemize}
    \end{block}
\end{frame}

\begin{frame}[fragile,t]
    \frametitle{test suite}

    \begin{columns}[t]
        \small
        \column{.45\textwidth}
        \begin{onlyenv}<1>
            \begin{block}{functional codes}
                \begin{lstlisting}[language=Python,columns=fullflexible]
class TestSuite:
  def run(self):
    result=TestResult()
    for x in self.tests:
      result.merge(x.run())
    return result
                \end{lstlisting}
            \end{block}
        \end{onlyenv}

        \begin{onlyenv}<2-3>
            \begin{block}{functional codes}
                \begin{lstlisting}[language=Python,columns=fullflexible]
class TestSuite:
  def run(self):
    result=TestResult()
    for x in self.tests:
      x.run(result)
    return result
                \end{lstlisting}
            \end{block}
        \end{onlyenv}

        \begin{onlyenv}<4->
            \begin{block}{functional codes}
                \begin{lstlisting}[language=Python,columns=fullflexible]
class TestSuite:
  def run(self,result):
    for x in self.tests:
      x.run(result)
    return result
                \end{lstlisting}
            \end{block}
        \end{onlyenv}

        \column{.45\textwidth}
        \begin{onlyenv}<1-2>
            \begin{block}{test case}
                \begin{lstlisting}[language=Python,columns=fullflexible]
class TestCase:
  def run(self):
    result=TestResult()
    result.started()
    self.setUp()
    try:
      method=getattr(self,self.name)
      method()
    except:
      result.failed()
    self.tearDown()
    return result
                \end{lstlisting}
            \end{block}
        \end{onlyenv}

        \begin{onlyenv}<3->
            \begin{block}{test case}
                \begin{lstlisting}[language=Python,columns=fullflexible]
class TestCase:
  def run(self,result):
    result.started()
    self.setUp()
    try:
      method=getattr(self,self.name)
      method()
    except:
      result.failed()
    self.tearDown()
                \end{lstlisting}
            \end{block}
        \end{onlyenv}
    \end{columns}
\end{frame}

\begin{frame}
    \frametitle{test suite}

    \begin{block}{To-Do}
        \begin{itemize}
            \item[$\surd$] Invoke test method
            \item[$\surd$] Invoke setUp first 
            \item[$\surd$] Invoke tearDown afterward 
            \item Invoke tearDown even if the test method fails 
            \item[$\surd$] Run multiple tests 
            \item[$\surd$] Report collected results 
            \item[$\surd$] Log string in WasRun 
            \item[$\surd$] Report failed results
            \item Catch and report setUp() and tearDown() errors
            \item Automatically create TestSuite() from a TestCase class
        \end{itemize}
    \end{block}
\end{frame}

\begin{frame}
    \frametitle{review}

    \begin{block}{~}
        \begin{itemize}
            \item 
                Don't go far away from green bars.
        \end{itemize}
    \end{block}
\end{frame}

\section{Summaries \& Experiences}

\subsection{summaries}

\begin{frame}[t]
  \frametitle{Summaries}

  \begin{block}{~}
    \begin{enumerate}
    \item 
      red $\xrightarrow{\text{functional code}}$ green $\xrightarrow{\text{refine}}$ green $\xrightarrow{\text{next feature}}$ red $\rightarrow$ ...
    \item<2->
      Get to a green bar asap.
      \begin{onlyenv}<2>
        \begin{itemize}
        \item
          Fake it, then make it.
        \item 
          Obvious implementation.
        \item 
          Keep the triangle.
        \end{itemize}
      \end{onlyenv}
    \item<3>
      Refine to the currently cleanest codes
      \begin{itemize}
      \item
        Keep green.
      \item 
        Small paces, always testing.
      \item
        NEVER introduce new feature while refining.
      \item 
        As clean as you can, even the codes which will be definitely discarded in final release.
      \item 
        Clean up test-case codes as well.
      \item 
        Manage your codes, by tools.
      \end{itemize}
    \end{enumerate}
  \end{block}
\end{frame}

\begin{frame}
    \frametitle{Why TDD works?}

    \begin{columns}[t]
        \column{.4\textwidth}
        \begin{block}{steps}
        \begin{enumerate}
        \item
            requirement analyze
        \item 
            design
        \item 
            code
        \item 
            test
        \item 
            release/deploy
        \end{enumerate}
        \end{block}

        \column{.5\textwidth}
        \begin{block}{CMMI: level 2}
            \begin{itemize}
                \item REQuirement Management
                \item Project Planning
                \item Project Monitoring and Control
                \item Supply Agreement Management 
                \item Product and Process Quality Assurance
                \item Configuration Management 
            \end{itemize}
        \end{block}
    \end{columns}

\end{frame}

\begin{frame}
    \frametitle{Why TDD works?}

    \begin{block}{Reducing failures}
    \begin{enumerate}
        \item
            quick feedback
        \item 
            clean design \& code
        \item 
            thinking in testing
        \item 
            be happy, be solid
    \end{enumerate}
    \end{block}

\end{frame}

\section{Q\&A}

\begin{frame}
  \frametitle{}
  \begin{center}
    \Huge
    Questions?
  \end{center}
\end{frame}

\begin{frame}
    \frametitle{Q\&A}

    \begin{block}{~}
        \begin{itemize}
        \item<1->
            Where shall I write to-do lists?
        \item<2->
            It takes too long to run regression\ldots
        \item<3->
            It takes too long to compile\ldots
        \item<4->
            I am working on untestable features
            \begin{itemize}
                \item
                    external systems: databases, file systems, \ldots
                \item
                    undeterminism: probability algorithms, parallel/distributed systems, \ldots
                \item 
                    third party codes and generated codes
                \item 
                    GUIs
                \item 
                    compiler
            \end{itemize}
        \item<5>
            I am tired\ldots
        \end{itemize}
    \end{block}
\end{frame}

\end{document}

